\documentclass[../DoAn.tex]{subfiles}
\begin{document}
Dựa trên những yêu chức năng và phi chức năng, kết hợp với kiến thức và kỹ năng cá nhân, em đã chọn ra các công nghệ phù hợp để xây dựng hệ thống, một vài công nghệ chính như: Docker,
Kubernetes, .NET 8, PostgreSQL, ReactJS.

\section{Docker}
\label{section:3.1}
Docker là một nền tảng mã nguồn mở cho phép các nhà phát triển xây dựng, triển khai, thực thi, cập nhật và quản lý các "container" -
là một khối thống nhất, được chuẩn hoá bao gồm mã nguồn, các thư viện hệ điều hành (Operating System) và các gói cần thiết để thực thi
mã nguồn đó trong mọi môi trường khác nhau.\cite{DockerOverview}

Các "container" đơn giản hóa việc phát triển và triển khai các ứng dụng phân tán. Chúng đã trở nên ngày càng phổ biến khi các tổ chức chuyển sang hướng phát triển hoàn toàn dựa trên
các dịch vụ đám mây (Cloud-native Development) và môi trường đa đám mây (Hybrid Multicloud Environment).
Các nhà phát triển có thể tạo ra một "container" mà không cần Docker, bằng cách làm việc trực tiếp với các chức năng tích hợp vào Linux và các hệ điều hành khác.
Tuy nhiên, Docker làm cho việc đó nhanh hơn, dễ dàng và an toàn hơn. Docker báo cáo hiện tại có hơn 13 triệu nhà phát triển sử dụng nền tảng này.\cite{DockerIBM}

Trong phạm vi dự án, Docker được sử dụng để đóng gói các thành phần trong hệ thống thành "container", như cơ sở dữ liệu, máy chủ xác thực người dùng, dịch vụ gửi thư điện tử,...
Nhờ đó, việc thực thi, kiểm thử mã nguồn không có sự khác biệt khi môi trường thay đổi: giữa máy tính cá nhân để phát triển, và thực tế trên máy chủ.
Đồng thời, hệ thống được đưa lên các máy chủ một cách đơn giản, tiết kiệm hơn, và không gặp phải những lỗi liên quan tới môi trường máy chủ, như hệ điều hành, xung đột giữa các gói và thư viện.
Ngoài ra, Docker cung cấp khả năng quản lý phiên bản tương tự như mã nguồn, chúng được sử dụng để theo dõi các thay đổi, đánh dấu phiên bản ổn định, và quay lại phiên bản cũ hơn
trong trường hợp có sự cố với phiên bản mới. Các "container" có thể được sử dụng như một bản thiết kế, giúp tiết kiệm thời gian và công sức trong việc
xây dựng các "container" mới dựa trên những gì đã có.

Hơn hết, việc xây dựng một quy trình kiểm thử, triển khai tự động cho hệ thống được đơn giản hoá, nhờ vào sự ổn định của các "container" và các công cụ hỗ trợ của Docker.
Điều này không chỉ có lợi cho việc phát triển hệ thống, mà còn giúp cho việc bảo trì, nâng cấp, và mở rộng hệ thống trở nên dễ dàng hơn, hệ thống được kiểm thử tự động cho mọi thay đổi
trước khi được đưa lên môi trường thực tế.

\section{Kubernetes}
\label{section:3.2}
Kubernetes (viết tắt là K8s) là một nền tảng quản lý "container" mã nguồn mở, lập lịch trình và tự động hoá việc triển khai, vận hành và mở rộng các ứng dụng được đóng gói
trong "container", cụ thể là kiến trúc Microservices. Nền tảng này được phát triển bởi Google với mục đích tối ưu hoá - tự động hoá nhiều quy trình phát triển và vận hành ứng dụng
(Development and Operations - DevOps) trước đây được thực hiện thủ công và đơn giản hóa công việc của các nhà phát triển phần mềm.\cite{KubernetesIBM}

Thực tế, Kubernetes là dự án phát triển nhanh nhất trong lịch sử phần mềm mã nguồn mở, sau Linux. Theo một nghiên cứu năm 2021 của Tổ chức điện toán hoạt động trên đám mây
(Cloud Native Computing Foundation - CNCF), số lượng kỹ sư sử dụng Kubernetes tăng 67\% lên 3,9 triệu người. Đó là 31\% tổng số nhà phát triển back-end,
tăng 4\% trong vòng một năm. Một số lượng lớn các công ty, tổ chức trong lĩnh vực công nghệ thông tin đã đưa Kubernetes vào để phát triển và vận hành các ứng dụng của mình.\cite{KubernetesIBM}

Trong phạm vi dự án, Kubernetes được sử dụng để quản lý toàn bộ các "container" trong hệ thống, đảm bảo các "container" luôn hoạt động ổn định và có thể mở rộng khi cần thiết.
Cụ thể hơn, các "container" luôn được Kubernetes kiểm tra trạng thái, nếu có lỗi xảy ra, Kubernetes sẽ tự động khởi động lại "container" đó,
hoặc khởi động lại toàn bộ hệ thống nếu cần thiết. Trong các trường hợp mà một hoặc một vài "container" có mức độ truy cập lớn, Kubernetes sẽ tự động gia tăng số lượng "container" đó
bằng cách tạo ra các bản sao, và phân phối lượng truy cập đến các bản sao đó nhằm tránh quá tải gây ra lỗi hệ thống. Ngoài ra, khi được triển khai lên các môi trường dịch
vụ đám mây, như Google Cloud Platform hay Microsoft Azure, Kubernetes có thể hoạt động đồng thời ở nhiều vùng địa lý khác nhau, giúp gia tăng tính sẵn sàng của hệ thống và giảm thiểu
độ trễ khi truy cập nhiều địa điểm khác nhau trên thế giới.

Hơn hết, Kubernetes cung cấp một khả năng tích hợp mạnh mẽ với các quy trình phát triển và vận hành ứng dụng tự động, giảm thiểu thời gian và công sức của nhà phát triển trong việc
liên tục đưa ra các thay đổi và cập nhật cho hệ thống thực tế. Cụ thể, khi mã nguồn có một sự thay đổi, Kubernetes cho phép cập nhật một thành phần cụ thể của hệ thống, mà không làm ảnh hưởng
đến các thành phần khác đang hoạt động, việc thay thế thành phần đó cũng chỉ được thực hiện sau khi quá trình kiểm thử và triển khai thành công. Điều này giúp cho hệ thống luôn ổn định,
giảm tối đa sự thời gian gián đoạn của hệ thống đang chạy thực tế.


\section{.NET 8 và ASP .NET Core}
\label{section:3.3}

.NET là một nền tảng phát triển phần mềm mã nguồn mở, miễn phí và đa nền tảng được phát triển bởi Microsoft. Nền tảng này cung cấp cho nhà phát triển
rất nhiều công cụ và thư viện để xây dựng ứng dụng trên đa dạng hệ điều hành, kiểu thiết bị, bao gồm phần mềm Desktop, ứng dụng di động, trò chơi điện tử,
ứng dụng IoT (Internet vạn vật - Internet Of Things), ứng dụng Web. Ngôn ngữ lập trình chủ yếu của .NET là C\#, được thiết kế bởi Microsoft hoàn toàn dựa trên
cơ sở lý thuyết và bám sát vào các khái niệm của lập trình hướng đối tượng hiện đại, giúp cho nhà phát triển dễ dàng làm quen và sử dụng thành thạo
ngôn ngữ này. Ngoài ra, .NET còn hỗ trợ các ngôn ngữ lập trình khác như F\#, Visual Basic, C++,... Phiên bản mới nhất của .NET là .NET 8, kèm theo
C\# 12.0 được ra mắt vào ngày 14 tháng 11 năm 2023 với nhiều cải tiến về hiệu suất thực thi và khả năng tương thích.\cite{DotNetMS}

Trong đó, ASP .NET Core là một framework được xây dựng  trên .NET với mục đích phát triển các ứng dụng Web hiệu suất cao, đa nền tảng,
và dễ dàng bảo trì, mở rộng. Đặc biệt, framework này được tối ưu hoá tốt để xây dựng các ứng dụng sử dụng "container" và triển khai chúng với
kiến trúc Microservices, cụ thể là các công cụ như Docker và Kubernetes.\cite{DotNetMS}

So sánh với các framework khác như Spring Boot của sử dụng ngôn ngữ Java, hay Django của sử dụng ngôn ngữ Python, ASP .NET Core có hiệu năng cao hơn,
và được hỗ trợ tốt hơn bởi cộng đồng nhà phát triển và Microsoft. Ngoài ra, mã nguồn của ASP .NET Core được áp dụng rất nhiều mẫu thiết kế phần mềm
hiện đại, giảm thiểu thời gian cài đặt thủ công của nhà phát triển để họ có thể tập trung vào các nghiệp vụ quan trọng hơn, ví dụ như:
Dependency Injection, Repository, Unit Of Work,...

Trong phạm vi dự án, các thành phần dịch vụ trong hệ thống được xây dựng hoàn toàn bằng ASP .NET Core, và các thư viện lớn như Entity Framework Core,
Hot Chocolate, MediatR, gRPC, RabbitMQ,... được sử dụng để giảm thiểu thời gian phát triển, và tăng tính ổn định của hệ thống.

Cụ thể, Entity Framework Core là một thư viện hỗ trợ truy cập cơ sở dữ liệu, giúp cho việc tương tác với cơ sở dữ liệu trở nên đơn giản hơn,
các bản thiết kế của thực thể được định nghĩa dưới dạng các lớp, và các truy vấn được thực hiện bằng các phương thức của các lớp đó. Các thay đổi
tới thiết kế cơ sở dữ liệu cũng được quản lý bằng mã nguồn ở tầng ứng dụng, bao gồm quản lý phiên bản, tự động áp dụng các thay đổi khi triển
khai hệ thống trên môi trường mới. Các truy vấn phức tạp cũng có thể được dịch ra và thực thi trên cơ sở dữ liệu một cách tối ưu thông qua các phương
thức được cung cấp sẵn bởi framework và .NET.

Ngoài kết nối tới cơ sở dữ liệu, các kết nối khác trong hệ thống cũng rất quan trọng và hoàn toàn có thể dễ dàng được thực hiện trong chính mã nguồn
ở tầng ứng dụng. Cụ thể hơn, kết nối giữa các dịch vụ trong kiến trúc Microservice thông qua giao thức gọi hàm từ xa (Remote Procedure Call - RPC) hay hàng đợi tin nhắn (Message queue),
kết nối giữa dịch vụ và tầng ứng dụng thông qua giao diện GraphQL API, RESTful API, hay Web Socket, đều được thực hiện bằng các thư viện hỗ trợ sẵn
trong ASP .NET Core.


\section{PostgreSQL}
\label{section:3.4}
PostgreSQL là một cơ sở dữ liệu mã nguồn mở có mức độ uy tín cao về tính tin cậy, linh hoạt, và hỗ trợ các tiêu chuẩn kỹ thuật mã nguồn mở.
Khác với các hệ quản trị cơ sở dữ liệu quan hệ khác (Relational Database Management Systems), PostgreSQL hỗ trợ cả các loại dữ liệu không quan hệ
và quan hệ. Ban đầu được phát triển vào năm 1986 như một phiên bản tiếp theo của INGRES (dự án cơ sở dữ liệu quan hệ SQL mã nguồn mở bắt đầu từ những năm 1970),
POSTGRES, hiện được biết đến với tên PostgreSQL, là ý tưởng của Michael Stonebraker, một giáo sư khoa học máy tính tại Berkeley. Năm 1994, PostgreSQL chính thức
được ra đời, và vẫn liên tục được phát triển và cập nhật phiên bản mới.\cite{PostgresIBM}

Về điểm mạnh, PostgreSQL có một kiến trúc mở, được kiểm chứng về tính bảo mật, tính toàn vẹn dữ liệu nhưng vẫn rất linh hoạt và dễ tiếp cận. Với tập
dữ liệu lớn, so với các đối thủ khác, hiệu suất đọc và ghi dữ liệu của PostgreSQL được ghi nhận là tốt hơn, các kỹ thuật tối ưu hoá truy vấn
cũng được hỗ trợ tốt hơn. Bên cạnh đó, PostgreSQL cũng hỗ trợ xử lý các truy vấn đồng thời (Concurrency) một cách an toàn mà không cần chặn
truy cập tới bản ghi (Record Locking), giúp tăng hiệu suất và giảm thiểu thời gian chờ đợi của các truy vấn thực thi song song.\cite{PostgresIBM}

Trong phạm vi dự án, PostgreSQL được sử dụng làm cơ sở dữ liệu chính cho hệ thống, lưu trữ các dữ liệu dài hạn bởi khả năng cung cấp hiệu suất cao
nhưng vẫn gọn nhẹ, tiết kiệm về bộ nhớ, khả năng tính toán của hệ thống. So sánh với các nền tảng khác, như Microsoft SQL Server, hay MySQL của Oracle,
PostgreSQL tương thích tốt hơn với Docker và Kubernetes do khả năng làm việc trên môi trường Linux tốt hơn, tiết kiệm tài nguyên xử lý hơn mà
vẫn đảm bảo các chức năng cơ bản của một hệ quản trị cơ sở dữ liệu quan hệ.

Ngoài ra, PostgreSQL cũng tương thích rất tốt với ASP .NET Core và Entity Framework Core, không có nhiều sự khác biệt so với Microsoft SQL Server
của chính Microsoft. Điều này giúp cho sử dụng PostgreSQL với phạm vi của dự án là một lựa chọn tốt về mặt kỹ thuật, tiết kiệm tài nguyên và chi
phí hơn so với các hệ quản trị cơ sở dữ liệu khác.

\section{ReactJS}
\label{section:3.5}
React JS là một thư viện JavaScript được phát triển bởi Facebook, được sử dụng để xây dựng giao diện người dùng cho các ứng dụng Web đơn trang
(Single Page Application). Thư viện này được đánh giá rất cao bởi cộng đồng nhà phát triển về hiệu suất, khả năng mở rộng, và khả năng tương thích với các thư viện khác.
Việc phát triển ứng dụng với React JS xoay quanh các thành phần (Component) và trạng thái (State), chúng được thiết kế để tái sử dụng, nhằm tạo ra những giao diện
người dùng phức tạp hơn.

Về điểm mạnh, React giúp tạo các giao diện người dùng tương tác một cách dễ dàng. Thiết kế các khung nhìn đơn giản cho từng trạng thái trong ứng dụng, sau đó, React sẽ cập nhật và
tạo ra đúng các thành phần phù hợp khi dữ liệu có sự thay đổi. Việc khai báo các khung nhìn tường minh sẽ khiến cho mã nguồn dễ sử dụng và gỡ lỗi hơn. Ngoài ra, do được
viết bởi JavaScript, thư viện này cho phép nhà phát triển thao tác với DOM (Document Object Model) và BOM (Browser Object Model) một cách dễ dàng.

So sánh với việc phát triển giao diện người dùng bằng mô hình MVC (Model-View-Controller) truyền thống, React JS có nhiều ưu điểm hơn. Cụ thể, người dùng có thể tương tác nhiều
hơn với giao diện người dùng, các thành phần có thể được tạo ra một cách độc lập, và có thể được sử dụng lại trong các trang khác nhau. Nhà phát triển cũng có thể dễ dàng
thực hiện các tính toán phức tạp ngay tại ứng dụng của mình, mà không cần phải gửi dữ liệu lên máy chủ, giúp giảm thiểu tải cho máy chủ và tăng hiệu suất cho ứng dụng.

Đối với các framework JavaScript khác như Angular hay VueJS, ReactJS có điểm mạnh là tính linh hoạt, tương thích cao với nhiều thư viện bên thứ ba. Lý do cho điều này là vì
React JS chỉ là một thư viện cung cấp những thành phần cơ bản nhất, đối với các yêu cầu phức tạp hơn, nhà phát triển có thể sử dụng các thư viện khác để mở rộng thêm mà không
bị giới hạn sự lựa chọn. Ngoài ra, nhà phát triển cũng không bị ràng buộc bởi các quy tắc, cấu trúc của framework, giúp cho việc xây dựng các tiêu chuẩn riêng của dự án
về mã nguồn được tự do và có chủ đích hơn.

Trong phạm vi dự án, React JS được sử dụng để xây dựng toàn bộ giao diện người dùng, tạo ra một ứng dụng Web đơn trang tách biệt hẳn với các dịch vụ trong hệ thống. Chúng
giao tiếp với nhau thông qua các API được cung cấp bởi các dịch vụ. Điều này giúp cho việc triển khai và kiểm thử trở nên độc lập với máy chủ, quá trình phát triển và bảo trì
ứng dụng cũng trở nên đơn giản hơn, tiết kiệm thời gian và chi phí.

\end{document}