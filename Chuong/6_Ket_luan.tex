\documentclass[../DoAn.tex]{subfiles}
\begin{document}
\section{Kết luận}
Đồ án "Xây dựng Hệ thống Quản lý dự án trên mô hình kiến trúc Microservices" đã được triển khai nhằm giải quyết bài toán đề ra ban đầu đó là
cung cấp một hệ thống quản lý dự án có những tính năng cơ bản, quan trọng nhất nhằm đáp ứng nhu cầu của một quy trình quản lý dự án.
Việc áp dụng kết quả đạt được sẽ giúp giảm thiểu gánh nặng trong việc quản lý dự án đối với cả người quản lý và cả thành viên trong các dự án.
Bằng việc sử dụng công nghệ phía backend là .NET 8, PostgreSQL, Docker, Kubernetes kết hợp với các công nghệ phía front-end là React JS, Ant Design,
hệ thống đầu ra có thể chạy một cách ổn định trên các nền tảng trình duyệt web thông dụng như Chrome, Firefox hay Safari, với giao diện được
thiết kế đơn giản dễ sử dụng. Thông qua hệ thống người dùng là quản lý dự án có thể lập kế hoạch cho dự án và điều phối công việc
một cách hiệu quả cũng như giám sát tiến độ của dự án một cách chặt chẽ. Đối với người dùng là thành viên trong các dự án thì
họ có thể kiểm soát được các công việc của mình trong từng dự án và sắp xếp công việc sao cho phù hợp với mức độ ưu tiên.

Trong quá trình hoàn thành đồ án tốt nghiệp, kết quả đạt được không chỉ dừng lại ở sản phẩm cuối cùng mà còn là những kiến thức và kỹ năng
cá nhân em đã tích lũy được, cũng như đúc kết được nhiều kinh nghiệm quý báu cho bản thân. Thông qua việc xây dựng hệ thống đã giúp em có
cơ hội được vận dụng, liên kết kiến thức của các môn học để giải quyết các vấn đề xuất hiện ở nhiều bước trong suốt vòng đời phát triển của
một phần mềm. Từ việc phân tích nghiệp vụ của bài toán, xác định yêu cầu hệ thống, thiết kế và xây dựng hệ thống hoàn chỉnh.
Ngoài ra em cũng học được kỹ năng nghiên cứu tài liệu, kỹ năng quản lý thời gian, lập kế hoạch và phân chia công việc một cách hiệu quả.
Những kiến thức trên đều là những kiến thức quan trọng, là hành trang cần thiết giúp em phát triển sự nghiệp sau này.

\section{Hướng phát triển}
Tuy đã cố gắng để hoàn thiện đồ án và đạt được mục tiêu ban đầu đề ra, nhưng trong thời gian làm đồ án có giới hạn và kiến thức còn hạn chế
nên hệ thống còn gặp phải nhiều vấn đề cần phải giải quyết. Việc tiếp tục phát triển sản phẩm để phù hợp khi áp dụng thực tế là vô cùng cần thiết.
Trong phần này, một số hướng phát triển cho hệ thống sẽ được đưa ra, mục tiêu là đem lại trải nghiệm sử dụng tốt hơn tới với người dùng với
những tính năng hiện có. Đồng thời đề xuất những tính năng mới có thể phát triển để hệ thống có thể đáp ứng được thêm các nhu cầu
cần thiết của người dùng.

Thứ nhất, về mặt giao diện , hệ thống hiện tại sử dụng một thư viện Ant Design cùng với Tailwind CSS và giao diện hệ thống đang dừng lại ở mức
sử dụng được và phù hợp với đại đa số người dùng. Tuy nhiên với xu hướng cá nhân hóa trong tương lai, việc có một số tính năng như cho phép
người dùng thay đổi giao diện hệ thống như về màu sắc, độ đậm nhạt font chữ, có thêm giao diện tối,... phù hợp với sở thích cá nhân là cần thiết.
Tiếp đến, hệ thống cần bổ sung thêm các unit test case, performance test case cũng như CI/CD giúp đảm bảo cho việc hệ thống hoạt động ổn định
với những thay đổi trong tương lai. Hiện tại hệ thống vẫn chưa hỗ trợ người dùng tùy chỉnh quá nhiều trong việc quản lý dự án,
trong tương lai nên thêm các thuộc tính và tính năng để người dùng có thể tự xây dựng quy trình làm việc phù hợp với từng dự án và
các loại mô hình phát triển phần mềm khác nhau. Tiếp đến, hệ thống cần có khả năng tích hợp với các hệ thống khác để tạo nên sự liên kết
cùng như linh hoạt cho người dùng. Cuối cùng là vấn đề báo cáo, trong tương lai hệ thống nên cung cấp thêm nhiều loại các báo cáo thống kê
chi tiết trực quan khác nhau phù hợp với nhiều loại hình dự án và đối tượng người dùng, giúp cho ban quản lý theo dõi được hiệu quả hoạt động
của các cá nhân trong dự án.

\end{document}