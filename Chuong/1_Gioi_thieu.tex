\documentclass[../DoAn.tex]{subfiles}
\begin{document}

\section{Đặt vấn đề}
\label{section:1.1}

Trong suốt quá trình phát triển ngành công nghệ thông tin, việc quản lý dự án một cách hiệu quả đã luôn là mục tiêu của rất nhiều tổ chức và cá nhân.
Quản lý dự án là một công việc rất quan trọng và đòi hỏi tinh thần làm việc nhóm cũng như khả năng tư duy logic khá cao. Bên cạnh đó, quản lý dự án luôn cần thiết đối với những dự án, kế hoạch, công trình quy mô lớn.

Quản lý dự án là một ngành học nghiên cứu về sự phối hợp và kiểm soát chặt chẽ từ việc lên kế hoạch dự án, quản lý thời gian, phân bổ nguồn lực và cuối cùng là phát triển dự án để đảm bảo dự án hoàn thành đúng hạn,
trong phạm vi ngân sách. Kết quả của dự án là đạt những yêu cầu về mặt kỹ thuật, chất lượng sản phẩm hay dịch vụ bằng tất cả những phương pháp và điều kiện tối ưu. Sau khi tìm hiểu về quản lý dự án là gì,
chúng ta cũng có thể hình dung được quá trình của công việc quản lý dự án. Gồm 3 giai đoạn chính là lập kế hoạch, điều phối thực hiện và giám sát tiến độ.

Lập kế hoạch là giai đoạn khởi đầu cho một dự án. Gắn liền với những ý tưởng bằng cách xây dựng mục tiêu, xác định vai trò của từng cá nhân, tính toán các nguồn lực tham gia và phối hợp thành một quá trình thống nhất, logic nhất.
Có thể lập kế hoạch qua sơ đồ hoặc qua các phương pháp truyền thống.

Điều phối thực hiện là sự phân phối các nguồn lực gồm có vốn, lao động, trang thiết bị. Từ đó sẽ có phương pháp giám sát dự án đảm bảo theo kịp tiến độ thời gian.
Phác thảo một sơ đồ gồm có ngày bắt đầu, ngày kết thúc và dự trù cả những tình huống xấu có thể xảy ra.

Giám sát tiến độ công việc là hành động của quá trình điều phối chính là giám sát. Nhiệm vụ chính của công đoạn này là phân tích tình hình, báo cáo tình trạng
và đề ra những biện pháp nếu có những trở ngại trong khi thi công. Song song với giám sát, có sự đánh giá khách quan kết quả giữa kỳ và cuối kỳ để rút kinh nghiệm hoặc thay đổi phương án.

Từ vấn đề thực tiễn nêu trên em quyết định lựa chọn đề tài "Xây dựng hệ thống quản lý phân công công việc và theo dõi tiến độ dự án" để đáp ứng nhu cầu cho việc quản lý dự án cho các doanh nghiệp hiện nay.

\newpage

\section{Mục tiêu và phạm vi đề tài}
\label{section:1.2}

Hiện nay trên thị trường có rất nhiều phần mềm quản lý dự án nổi bật như là Asana, Azure Devops hay Jira. Mỗi phần mềm đều có những ưu và nhược điểm khác nhau.
Ví dụ như Azure Devops nhiều tính năng nhưng khó sử dụng, thời gian cần để làm quen lâu, chỉ phù hợp với các dự án lớn, cần nhiều tính năng chuyên sâu.
Mặt khác, Asana thì dễ sử dụng nhưng chỉ phù hợp các dự án và đội nhóm nhỏ, phù hợp với các dự án cần tích hợp nhiều kênh thông tin bên thứ ba.

Chính vì vậy, đồ án đưa ra mục tiêu là xây dựng một phần mềm quản lý dự án có tính linh hoạt, phù hợp với nhiều mô hình phát triển phần mềm khác nhau, có thể đáp ứng được các nhu cầu quản lý dự án,
phân công và theo dõi tiến độ của công việc.

\section{Định hướng giải pháp}
\label{section:1.3}

Giải pháp đưa ra là hướng tới xây dựng một hệ thống dựa trên các nguyên tắc cơ bản của các mô hình phát triển phần mềm hiện đại như Scrum, Kanban, Agile.

\section{Bố cục đồ án}
\label{section:1.4}

Trong phần tiếp theo của đồ án tốt nghiệp, em xin trình bày các nội dung chính của đồ án bởi các chương như sau:

\textbf{Chương 2.} Khảo sát và Phân tích yêu cầu: Thực hiện việc khảo sát hiện trạng và phân tích yêu cầu hệ thống.

\textbf{Chương 3.} Công nghệ sử dụng: Giới thiệu về các công nghệ sử dụng trong đồ án.

\textbf{Chương 4.} Thiết kế và xây dựng hệ thống: Trình bày về việc thiết kế và xây dựng hệ thống.

\textbf{Chương 5.} Các giải pháp và đóng góp nổi bật: Trình bày về các giải pháp và đóng góp nổi bật của đồ án.

\textbf{Chương 6.} Kết luận và hướng phát triển: Kết luận chung về đồ án tốt nghiệp, và đề ra hướng phát triển mở rộng trong tương lai.

\end{document}