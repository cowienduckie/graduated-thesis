\documentclass[../DoAn.tex]{subfiles}
\begin{document}

\begin{center}
    \Large{\textbf{TÓM TẮT NỘI DUNG ĐỒ ÁN}}\\
\end{center}
\vspace{1cm}

Quản lý dự án và công việc của các thành viên luôn là vấn đề chúng ta gặp phải trong suốt quá trình làm việc. Tại một thời điểm thì một nhân viên có thể tham gia nhiều hơn một dự án cùng lúc và mỗi dự án thì thường sẽ có từ bốn thành viên trở lên có đôi khi sẽ có nhiều thành viên tham gia hơn. Khi có quá nhiều việc cần phải làm bạn không thể quản lý nổi những việc mình cần làm là việc gì, nên ưu tiên công việc nào trước, những công việc nào con đang dang dở và rất nhiều vấn đề khác có thể gặp phải.

Đồ án được thực hiện với mục đích giải quyết các vấn đề trên là xây dựng một hệ thống giúp quản lý dự án, phân công công việc và theo dõi tiến độ dự án. Thông qua việc sử dụng hệ thống có thể giúp cho người dùng có thể quản lý các dự án mà mình đang tham gia, các công việc được giao trong từng dự án cũng như mức độ ưu tiên và trạng thái của từng công việc. Đối với những người có cương vị là người quản lý hoặc trưởng nhóm có thể theo dõi tiến độ của các dự án và phân chia công việc cho từng thành viên một cách hợp lý nhất.

Kết quả đạt được cuối cùng của đồ án là một hệ thống đáp ứng các nhu cầu đã kể trên cũng như sự phát triển của bản thân về các kỹ năng lập trình và phân tích giải quyết vấn đề.

\begin{flushright}
    Sinh viên thực hiện\\
    \begin{tabular}{@{}c@{}}
        \textit{(Ký và ghi rõ họ tên)}
    \end{tabular}
\end{flushright}

\end{document}